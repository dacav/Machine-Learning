As mentioned in Subsection~\ref{sub:Computational-steps}, I wrote two
different programs, one for each of the phases described in
Section~\ref{sec:Approaching-the-problem}.

\subsection{ Programs usage }

    The program \FileName{genes-rank} requires as parameters the path
    of the file containing the dataset and a name for the output file:
    \begin{verbatim}
    $ ./genes-rank ../dataset_1_2_3 ../results/thresholds.txt
    \end{verbatim}

    After the elaboration phase, it produces as output a text-file
    containing a list of genes sorted by the maximal information gain, as
    described in Paragraph~\ref{subsub:Ranking-the-genes}. The file
    produced by \FileName{genes-rank} contains rows like the following:
    \begin{verbatim}
    Gene 2376 : 154.5 0.698437155782
    Gene 1335 : 312.5 0.693746194896
    Gene 3544 : 994.0 0.693746194896
    Gene 1374 : 1419.5 0.681680475648
    ...
    \end{verbatim}

    Which can be matched by the \emph{regular expression}
    \begin{verbatim}
    Gene (\d+) : (-?\d+\.?\d*) (-?\d+\.?\d*)
    \end{verbatim}

    The first numeric value is the index of the gene with respect to the
    dataset file, while the second and the third are respectively the
    threshold and the information gain for that gene.

    This output file can be parsed by the \FileName{genes-clust} program,
    which can produce a clustering for genes according to what explained
    in Subsection~\ref{sub:Clustering-of-genes}. All the listed distance
    measure between clusters are available as options for the program:
    \begin{verbatim}
    $ ./genes-clust --help
    Usage: genes-clust [options] <vectors file>

    Available algorithms
            nearest-neighbor
            dist-average
            farthest-neighbor
            average-distance
            all-of-them

    Options:
      -h, --help            show this help message and exit
      --algorithm=ALGS      Enable distance algorithm
      --log-file=LOG_FILE   Target file for statistics
      -D DATASET, --dataset=DATASET
                            Dataset file
      -R GENES_RANK, --genes-rank=GENES_RANK
                            Dataset file
      -n TAKEBEST           Number of genes to take
    \end{verbatim}

    An usage example is the following:
%    \begin{verbatim}
%    $ ./genes-clust 
%    \end{verbatim}

\subsection{ Programs structure }
