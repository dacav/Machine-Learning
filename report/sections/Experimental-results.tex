\Image{pics/squared-errs}
      {Comparison between clustering quality by using different clustering
       techniques. The clustering has been run with 100 experiments. Each
       iteration merges two clusters (thus we have 100 iterations)}
      {img:Squared-errors}
      {\textwidth}

For each distance measure and for each iteration of the clustering
mechanism, the \FileName{genes-clust} program records the \emph{sum of squared errors} inside the log file.

I quckly wrote the \FileName{parse-log} program to parse it out and I fed
\emph{gnuplot} with the output, producing the graphs shown in
Figure~\ref{img:Squared-errors}.

The plot highlights that the quality of the clustering with different
distance measure is comparable, although the \emph{farthest neighbor} and
the \emph{average distance} techniques seems to work slightly better than
the other.
