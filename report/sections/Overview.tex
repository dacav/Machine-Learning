\subsection{ Objective }

    The project aimed to achieve a clustering among a given set of genes,
    basing on their expression level. I was provided with a dataset of 72
    records for patients affected either by \emph{Acute Lymphoblastic
    Leukemia} or by \emph{Acute Myeloid Leukemia}. Each record of the
    given dataset reports the expression levels for 5147 genes.

    For some genes, the expression level is a good index for
    discrimination of the two forms of leukemia: basing on how the dataset
    is partitioned (more details on this later) genes can be ranked
    according to their precision in splitting the dataset.

    My final objective was the hierarchical clustering of the genes having
    the best splitting performances, based on the similarity in the genes
    behavior.

\subsection{ The dataset }

    The provided dataset is a plain text file with the following format:
    \begin{itemize}
    \item   A row containing a \emph{tab-separated} list of genes names;
    \item   A row containing, for each gene, the kind of data represented
            for it. In this case each feature is \emph{continue} (i.e. a
            real number);
    \item   An empty row;
    \item   A row for each of the analyzed patients,
            having a \emph{tab-separated} list of expression levels one
            for each of the genes listed in the first row of the
            file. The last column of each patient reports the form of
            leukemia the patient is affected by.
    \end{itemize}

    As we are considering two forms of leukemia, we are in a binary
    classification setting.

\subsection{ Computational steps }

    The available data have been elaborated in two phases:
    \begin{enumerate}
    \item   The recognition of the best splitting threshold for each gene
             (described in Subsection~\ref{sub:Thresholds});
    \item   The clustering of the more expressive genes basing on how they
            classify the given dataset (described in
            Subsection~\ref{sub:Clustering}).
    \end{enumerate}

    In order to achieve a quick development, the \emph{Python} programming
    language has been used. I wrote two main programs according to the two
    elaboration phases required: \FileName{genes\_rank} and
    \FileName{genes\_clust}. The programs are described in
    Section~\ref{sec:Programs}


