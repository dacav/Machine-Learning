In this section the problem will be analyzed from a theoretical
perspective.

\subsection{ Recognizing Thresholds } \label{sub:Thresholds}

    Some genes can be more related than others with the kind of leukemia
    the patient is affected by: for a gene having a good correlation, we
    can infer the classification basing on the expression level, thus
    classifying differently an instance if the value is above or below a
    certain threshold.

    Supposing we already have a threshold for each gene, since we are in a
    \emph{supervised learning} setting, we could determine the
    aforementioned degree of correlation by measuring how good is the
    threshold-based prediction with respect to the real assignment
    of examples.

    First of all, however, we should determine for each gene the best
    threshold for the binary classification.

    \subsubsection{ Entropy of the Dataset }

        If we are given with a certain dataset, which is partitioned
        according to a set of classes $y_1 \cdots y_c$, we can use the
        \emph{Information Entropy} as index of uniformity of the dataset.

        The Entropy is a measure of the uncertainty of a random variable.
        For a discrete random variable $X$ taking values $\Set{ x_1 \cdots
        x_n }$ with a certain probability mass function $p$, the Entropy
        is defined as:
        \[
        H(X) = \sum_{i = 0}^n p(x_i) \cdot \log_2{p(x_i)}
        \]
        By construction, the Entropy is maximum for a random variable
        having uniform distribution, while it gets lower values for
        non-homogeneous distributions.

        We can consider a dataset $D$ as a set of values drawn from a
        categorical distribution, for which we can estimate the
        probability distribution among the classes $y_1 \cdots y_c$ as the
        proportion between the classes cardinality $D_1 \cdots D_c$ and
        the whole set cardinality. In other words we can define:
        \[
        \Prob{x \in y_i} = \frac{ \Card{D_i} }{ \Card{D} }
        \]

        In our case we need a binary classification, thus examples are
        partitioned in two subsets, $D_{\tiny{\AML}}$ and
        $D_{\tiny{\ALL}}$, of the dataset $D$. As we have a mass function
        we can compute the Entropy of the dataset.

    \subsubsection{ Computing the threshold }

        The less is the Entropy of a set, the least is the information we
        gain from knowing an example to belong that set. So far, when
        splitting a set $D$ into partitions $D_1 \cdots D_N$, we obtain a
        change in the overall Entropy: this is the concept of Information
        Gain:
        \[
        IG(D, D_1 \cdots D_N) =
            H(D) - \sum_{i = 0}^N \frac{ \Card{D_i} }{ \Card{D} } H(D_i)
        \]

        For each gene $g$, we search an expression level threshold $t_g$
        to be used as boundary for our dataset partitioning, thus
        splitting $D$ into two subsets:
        \[
        D_0 \:=\: \Set{ x \in D \:|\: x_g < t_g }
        \qquad
        D_1 \:=\: \Set{ x \in D \:|\: x_g \geq t_g }
        \]
        Once we have a partitioning, we can compute the Information Gain
        coming from it.

        Let $L = \Tuple{\Tuple{x_0, y_0} \cdots \Tuple{x_m, y_m}}$ be a
        list of pairs sorted by the first field. Each element of the list
        corresponds to an element of the dataset, and it's a pair
        $\Tuple{x,y}$ such that $x$ is the expression level of the gene
        $g$ and $y$ it's the label of the example. The candidate
        thresholds are the values
        \[
        T = \Set{ \frac{x_0 + x_1}{2} \:|\:
                  \Tuple{x_0, y_0} = L_i \:\land\:
                  \Tuple{x_1, y_1} = L_{i + 1} \:\land\:
                  y_0 \neq y_1 }
        \]

        The best suited threshold $t_g$ for a gene $g$ is the element of
        $T$ which maximizes the Information Gain coming from the splitting.

    \subsubsection{ Ranking the genes }

        As we computed a threshold for each gene, we have as many binary
        classifiers as the number of analyzed genes. Some of them may
        have a bad quality, while other may be better than other at
        fitting the dataset labeling. The ones showing better performances
        in splitting are likely to be more related with the leukemia type
        with respect to the other.

        The genes can be sorted according to the Information Gain they
        yield if used as splitting parameter.

\subsection{ Clustering of Genes } \label{sub:Clustering}

    In the \emph{threshold recognition} phase we exploited the examples
    labels to identify the genes showing a categorization capability with
    respect to our classes.

    In this phase we want to group together genes which show a similar
    behavior in terms of categorization of examples. In order to do this
    we rely on the theoretical concepts described in this section.

    \subsubsection{ Distance between clusters }

        In a $n$-dimensional vector space \VecFn, we can give
        many definitions of \emph{distance}. The most intuitive one
        is the \emph{euclidean distance}:
        \[
        \mbox{let } v, w \in \VecFn \qquad
        \Dist{v}{w} = \sqrt{\sum_{i = 1}^n (v_i - w_i)^2 }
        \]

        As we can compute the distance between two vectors, we can also
        compute the distance between two \emph{groups} of vectors
        Let $C = \Set{ c_1 \cdots c_n }$ and $D = \Set{ d_1 \cdots d_m }$
        be two clusters we want to compare. The following techniques can
        be used:
        \begin{itemize}
        \item   $\ClustDist{C}{D} = \Dist{\Avg{C}}{\Avg{D}}$;
        \item   $\ClustDist{C}{D} = \Min{\Dist{x}{y} \st \Tuple{x,y} \in
                                         C \times D}{}$;
        \item   $\ClustDist{C}{D} = \Max{\Dist{x}{y} \st \Tuple{x,y} \in
                                         C \times D}{}$;
        \item   $\ClustDist{C}{D} = \Avg{\Dist{x}{y} \st \Tuple{x,y} \in
                                         C \times D}{}$.
        \end{itemize}

        The clustering technique consists in an iterative process for
        which the pair of clusters having the minimum distance gets merged
        into a single set.

    \subsubsection{ Translating genes into vectors }

        In the previous phase we managed to define a binary classification
        threshold for each gene and find the genes which show a certain
        correlation with the disease. Now we want to group together genes
        which show the same way of classifying the dataset.

